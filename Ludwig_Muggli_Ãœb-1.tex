\documentclass{article}

% Die UTF-8 Kodierung ermöglicht Umlaute und Sonderzeichen.
\usepackage[utf8]{inputenc}
%\usepackage[T1]{fontenc}

% Wir laden und wählen Deutsch als Sprache (neue Rechtschreibung).
% Dadurch werden z.B. Deutsche Trennungsregeln verwendet.
\usepackage[ngerman]{babel}
\selectlanguage{ngerman}

% Wir wechseln zum dd.mm.yyyy Datumsformat.
\usepackage[ngerman]{isodate}
\date{\numdate\today}

% Einige mathematische Werkzeuge.
% 	- amsmath stellt essentielle mathematische Umgebungen wie z.B. equation.
%	- amsfonts enthält übliche mathematische Typographien.
%	- amssymb definiert viele mathematische Symbole.
% Tatsächlich lädt amssymb bereits intern das Package amsfont, d.h. zweiteres
% hätte hier nicht explizit geladen werden müssen.
\usepackage{amsmath, amsfonts, amssymb}

% Für kommutative Diagramme.
\usepackage{tikz-cd}

\usepackage{pgfplots}
\pgfplotsset{compat=1.17}

% Einige Tools für die Korrektur.
\usepackage{geometry}
\usepackage{marginnote}
\usepackage{xcolor}
\usepackage{pdfcomment}
\definecolor{color_annotation}{RGB}{255,50,50}	% Korrekturfarbe.
\newcommand{\highlight}[1]{\pdfmarkupcomment[markup = Highlight, color = yellow]{#1}~}
\newcommand{\strike}[1]{\pdfmarkupcomment[markup = StrikeOut, color = color_annotation]{#1}~}
\newcommand{\squiggle}[1]{\pdfmarkupcomment[markup = Squiggly, color = color_annotation]{#1}~}
\newcommand{\uline}[1]{\pdfmarkupcomment[markup = Underline, color = color_annotation]{#1}~}
\newcommand{\sidecomment}[1]{\marginnote{{\color{color_annotation}#1}}}
\newcommand{\comment}[1]{{\color{color_annotation} #1}}

\usepackage{enumitem}
\usepackage{titlesec}

\setlist[itemize]{noitemsep, topsep=0pt}
\setlength\parindent{0pt}
\titleformat{\subsection}[runin]
{\normalfont\large\bfseries}{\thesubsection}{1em}{}
\titleformat{\subsubsection}[runin]
{\normalfont\normalsize\bfseries}{\thesubsubsection}{1em}{}



\title {Übungsblatt 1 \\ \large{Optimierung I}}
\author{Ludwig Muggli \\ ludwig.muggli@outlook.de \\ Mtr. Nr. 1638691
        %\and
        %Erika Mustermann \\ erika.mustermann@web.de \\Mtr. Nr. 87654321
        }

\begin{document}

% Wir setzen den Titel etwas höher an.
\vspace*{-3cm}
{\let\newpage\relax\maketitle}

\subsection*{Aufgabe 1} Modelliere die gegebene Situation als ganzzahliges lineares Programm, wenn Alice und Bob den Gesamtnutzen maximieren wollen. Gib dazu - jeweils mit Erklärung - die verwendeten Variablen Restriktionen und die Zielfunktion an. \par \smallskip

In dem im Rahmen von Aufgabe 4 erstellten Dokuments \textit{Ludwig\_Muggli\_Blatt-1\_Aufgabe-1\_Aufgabe-4.zpl} finden sich detaillierte Anmerkungen, die der Beantwortung dieser Frage genüge tun sollten.

\subsection*{Aufgabe 2} Finden Sie ähnlich zum Beispiel aus dem Vorlesungsvideo jeweils ein Minimierungsproblem
\begin{center}
\begin{tabular}{ll}
Minimiere & $f(x)$ \\
u.d.N. & $x \in K \neq \emptyset$\\
\end{tabular}
\end{center}
welches keine Optimallösung hat, wenn
\begin{enumerate}[label=(\alph*)]
\item $K$ Kompakt ist \par \smallskip
Wähle hierzu $K = [0,1] \subset \mathbb{R}$ und $f: \mathbb{R} \to \mathbb{R}, \; x \mapsto \ln{x}$. Dann existieren keine optimalen Lösungen für das Problem \begin{center}
\begin{tabular}{ll}
Minimiere & $f(x)$ \\
u.d.N. & $x \in K \neq \emptyset$\\
\end{tabular}
\end{center}
denn für alle $x \in K$ existiert ein $\tilde{x} \in K$ mit $f(\tilde{x}) \leq f(x)$, da $\lim\limits_{x\to \infty}{f(x)} = - \infty$.
\item $f$ stetig und $K$ abgeschlossen ist.\par \smallskip
Wähle hierzu $K = [0, \infty) \subset \mathbb{R}$ und $f: \mathbb{R} \to \mathbb{R}, \; x \mapsto e^{-x}$. Sicherlich ist $K$ abgeschlossen, da das Komplement $K^C = (-\infty,0)$ offen ist.\newline
Es existieren keine optimalen Lösungen für das Problem
\begin{center}
\begin{tabular}{ll}
Minimiere & $f(x)$ \\
u.d.N. & $x \in K \neq \emptyset$\\
\end{tabular}
\end{center}
denn für alle $x \in K$ existiert ein $\tilde{x} \in K$ mit $f(\tilde{x}) \leq f(x)$, da $\lim\limits_{x \to \infty}{f(x)} = 0$ und $f(x) > 0$ f.a. $x \in K$.
\end{enumerate}
\subsection*{Aufgabe 3} Gegeben ist der folgende Zulässigkeitsbereich $X$:
\begin{align*}
x_1 &\geq 0\\
x_2 &\geq 0\\
x_2 &\leq 2\\
(x_2-1)^2 &\leq x_1\\
(x_2 -1)^2 &\geq x_1-2
\end{align*}
Die Funktion $f(x) = \frac{1}{(x_1+1)(x_2+1)}$ nimmt ihr Maximum auf einer konvexen Menge $K$ im ersten Quadranten an einem Extremalpunkt von $K$ an (dies muss nicht gezeigt werden). Nutze diesen Hinweis, um alle Punkte aus $X$ anzugeben, bei denen die Funktion $f$ auf $X$ ihr Maximum annimmt. \par \medskip

Man sieht leicht ein, dass die Extremalpunkte von $X$ genau jene rot markierten sind:
\begin{center}
\begin{tikzpicture}
\begin{axis}[
axis lines = center,
xlabel = $x_1$,
ylabel = $x_2$,
xmin = -0.5, xmax = 3.5,
ymin = -0.5, ymax = 2.5,
]
\addplot[samples=50,red,thick,fill=lightgray,opacity=0.8,domain=0:2] ((x-1)^2,x);
\draw (1,0)[fill=lightgray,opacity=0.8,draw=none] rectangle (2.99,2);
\addplot[samples=50,thick,fill=white,domain=0:2] ((x-1)^2+2,x);
\draw[thick,fill] (axis cs:1,2) -- (axis cs:3,2);
\draw[thick] (axis cs:1,0) -- (axis cs:3,0);
\node[label={center:{\Large$X$}},circle,mark=none] at (axis cs:1,1) {};
\node[label={0:{$(0,1)$}},circle,fill,inner sep=2pt,red] at (axis cs:0,1) {};
\node[label={90:{$(3,0)$}},circle,fill,inner sep=2pt,red] at (axis cs:3,0) {};
\node[label={270:{$(1,2)$}},circle,fill,inner sep=2pt,red] at (axis cs:1,2) {};
\node[label={270:{$(3,2)$}},circle,fill,inner sep=2pt,red] at (axis cs:3,2) {};
\node[label={90:{$(1,0)$}},circle,fill,inner sep=2pt,red] at (axis cs:1,0) {};
\end{axis};
\end{tikzpicture}
\end{center}
Damit kommen nur die Punkte $(3,2)$ und $(3,0)$ sowie Punkte in
$$\{(x_1,x_2) \in \mathbb{R}^n \mid (x_2-1)^2 = x_1, 0 \leq x_1, 0 \leq x_2 \leq 2\} = \{((x_2-1)^2,x_2) \in \mathbb{R}^n \mid 0 \leq x_2 \leq 2 \}$$
als Maxima der Funktion $f$ auf $X$ in Frage. Da
$$f(1,0) = \frac{1}{2} > \frac{1}{4} = f(3,0) > \frac{1}{12}=f(3,2)$$ scheiden $(3,2)$ und $(3,0)$ als mögliche Maxima der Funktion $f$ auf $X$ bereits aus. Es muss also von der Form $\{((x_2-1)^2,x_2) \in \mathbb{R}^n \mid 0 \leq x_2 \leq 2 \}$ sein.\par \smallskip
Dazu betrachten wir die erste Ableitung von $f$ im Punkt $((x_2-1)^2,x_2)$:
$$f'((x_2-1)^2,x_2) = \frac{x_2(3x_2-2)}{(x_2+1)^2(x_2^2-2x_2+2)^2}$$
Diese ist $0$ gdw. $x_2= 0$ oder $x_2 = \frac{2}{3}$. Daher können sich nur noch in den Punkten $(1,0), \left (\frac{1}{9},\frac{2}{3} \right)$ sowie an den Randpunkten $(1,0),(1,2)$ globale Minima befinden. \par \smallskip
Nun gilt
$$f(1,2) = \frac{1}{8} < f(1,0) = \frac{1}{2} < f \left (\frac{1}{9},\frac{2}{3} \right)=\frac{27}{50}$$

Damit nimmt die Funktion $f$ ihr Maximum auf $X$ im Punkt $\left (\frac{1}{9},\frac{2}{3} \right)$ an. \hfill $\square$

\subsection*{Aufgabe 4} Verwende ZIMPL um mit Hilfe deiner Modellierung aus Aufgabe 1 eine Optimallösung des dortigen Maximierungsproblems zu finden. \par \smallskip

Siehe hierzu Anlage \textit{Ludwig\_Muggli\_Blatt-1\_Aufgabe-1\_Aufgabe-4.zpl}.



\end{document}
